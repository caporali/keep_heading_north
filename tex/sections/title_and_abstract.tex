\begin{minipage}{0.99\textwidth}
	\definetitle{Text based game on a graph}{Francesco Caporali}
	\maketitle
	\defineabstract{%
		Text-based games represent one of the most primitive forms of electronic games. 
		The idea behind these games consists in making storyline advancements and game choices through (more or less complex) textual input from a player,
		schematisable as choices from a predefined set of alteratives.
		The aim of this project is to develop a working example of a text-based game using graphs as game maps.
		
		\noindent The game is structured as a dungeon in which you have to travel with the final aim of surviving until you reach the exit. 
		In the initial phase of the adventure the player will be assigned a fixed amount of energy and life points. 
		At each turn he/she will have to move around the game map consuming energy and potentially encountering hostile monsters to fight while trying not to lose life points.
		The ultimate goal is to reach the exit of the dungeon before having exhausted life or energy.
		% During the exploration one will be able to collect items, including treasures and consumables (which allows one to obtain game hints and clues).

		\noindent The generation of a game session involves the random generation of a weighted graph and then the distribution of monster objects on it. 
		This is followed by an automatic calculation of the minimum energy and life points required to be able to finish the game, varying the maximum number of monsters encountered		
		(by means of an independently developed algorithm that exploits shortest path algorithms).
		Finally, in accordance with a user selectable \textit{difficulty level} variable, the initial energy and life points will be assigned on the basis of the costs we have just calculated.
		% At this point, items will be distributed on the game map (still algorithmically).
	}
\end{minipage}